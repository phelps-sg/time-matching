%%This is a very basic article template.
%%There is just one section and two subsections.
\RequirePackage{lineno} 
\documentclass[12pt]{article}
%\usepackage[inner=2cm,outer=2cm]{geometry}
\usepackage{graphicx}
\usepackage{amsmath}
\usepackage{amsfonts}
\usepackage{subfigure}
\usepackage{verbatim}
\usepackage{xspace}
\usepackage{amscd}
\usepackage{rotating}
\usepackage{latexsym}
\usepackage{multicol}
\usepackage{array}
\usepackage{algorithm}
\usepackage{subfigure}
\usepackage[numbers,sort&compress]{natbib} 
%\usepackage{apacite}
\usepackage{url}
\usepackage{times}
\usepackage{setspace}

\usepackage{endnotes}
\let\footnote=\endnote

\topmargin 0.0cm
\oddsidemargin 0.2cm
\textwidth 16cm 
\textheight 21cm
\footskip 1.0cm

\begin{document}

\linenumbers

\baselineskip24pt

\author{%%%% Author details
Steve Phelps$^{1}$, Wing Lon Ng$^{2}$, Mirco Musolesi$^{3}$ and Yvan I. Russell$^{4,5}$\\
%%%%%%%%% Insert author address here
\normalsize{$^{1}$Department of Informatics, King's College London, UK.}\\
\normalsize{$^{2}$Bounded-Rationality Advancement in Computational Intelligence Laboratory, UK.}\\
\normalsize{$^{3}$Department of Geography, University College London, UK.}\\
\normalsize{$^{4}$Department of Psychology, Middlesex University, UK.} \\
\normalsize{$^{5}$CRC Evolution of Social Behaviour, University of G\"{o}ttingen, Germany.}
}
% 
% %%%% Subject entries to be placed here %%%%
% \subject{Behaviour (Biology), Psychology (Biology)}
% 
% %%%% Keyword entries to be placed here %%%%
% \keywords{Chimpanzees, Allogrooming, Reciprocity, Time matching}
% 
% %%%% Insert corresponding author and its email address}
% \corres{Steve Phelps\\
% \email{steve.phelps@kcl.ac.uk}}
% 
\title{Supplementary material for ``
Precise time-matching in chimpanzee allogrooming does not occur after a short delay''}

\section{Alternative event-pairing results}

Our event-pairing procedure pairs events together within a dyad whenever the 
direction of grooming changes, and we pair together the most recent event before 
the direction change.  During the review process, Michio Nakamura correctly 
pointed out that this is an arbitrary decision, and that we could equally well 
have chosen, e.g. the earliest event in the same direction.  In light of this comment,
we reran our analysis with this alternative pairing procedure.  The results
are presented are below, using plots in an identical format to the original 
manuscript.  As can be seen, the alternative pairing procedure does not
alter our conclusion that time-matching does not occur after a delay.

\begin{figure*}
 \begin{center}
  \includegraphics[scale=0.8]{figs/earliest-pairing/delay-histogram-crop.pdf}
  \caption{Histogram of $\Delta$ measured in 
minutes.\label{fig:delay-histogram}}
 \end{center}
\end{figure*}

\begin{figure*}
 \begin{center}
  \includegraphics[scale=0.55]{figs/earliest-pairing/reciprocity-by-treatment-crop.pdf}
  \caption{Box plots for the reciprocity measure $\rho  = |X-Y|/(X+Y)$ by 
comparison condition. Delayed grooming $\Delta \geq 0$ is shown on the left 
($n=555$), and within-bout grooming $\Delta < 0$ on the right ($n=277$).
  \label{fig:boxplots-by-condition}}
 \end{center}
\end{figure*}


\begin{figure*}
%   \begin{center}
    \subfigure[Reciprocity by chimpanzee - $\Delta < 0$]{
      \includegraphics[scale=0.46]{figs/earliest-pairing/abs-reciprocity-by-chimp-immediate-crop.pdf}
      \label{fig:abs-reciprocity-by-chimp:immediate}
    }
    \subfigure[Reciprocity by chimpanzee - $\Delta \geq 0$]{
      \includegraphics[scale=0.46]{figs/earliest-pairing/abs-reciprocity-by-chimp-delayed-crop.pdf}
      \label{fig:abs-reciprocity-by-chimp:delayed}
    } \\
    \centering
    \subfigure[Number of paired events by chimpanzee]{
      \includegraphics[scale=0.35]{figs/earliest-pairing/counts-bar-plot.pdf}
      \label{fig:counts-bar-plot}
    }
  \caption{Box plots for the reciprocity measure $\rho = |X-Y|/(X+Y)$ grouped by individuals.  
  Figure~a) above is restricted to within-bout grooming --- i.e. $\Delta < 0$ ---
  whereas b) below illustrates the delayed case $\Delta \geq 0$.  The corresponding
  sample sizes are summarised underneath in c).}.
  \label{fig:abs-reciprocity-by-chimp}
\end{figure*}

\begin{figure*}
  \begin{center}
    \subfigure[Reciprocity by dyad - $\Delta < 0$]{
      \includegraphics[scale=0.6]{figs/earliest-pairing/reciprocity-by-dyad-immediate-crop.pdf}
      \label{fig:reciprocity-by-dyad:immediate}
    }
    \subfigure[Reciprocity by dyad - $\Delta \geq 0$]{
      \includegraphics[scale=0.6]{figs/earliest-pairing/reciprocity-by-dyad-delayed-crop.pdf}
      \label{fig:reciprocity-by-dyad:delayed}
    }
  \end{center}
  \caption{Box plots for the reciprocity measure $\rho$ grouped by dyad.  
  Figure a) above is restricted to within-bout grooming --- i.e. $\Delta < 0$ ---
  whereas b) below illustrates the delayed case $\Delta \geq 0$.}
  \label{fig:reciprocity-by-dyad}
\end{figure*}



\begin{figure*}
\begin{center}
\subfigure[Time-matching (all)]{  \label{fig:regression:all}
\includegraphics[scale=0.43]{figs/earliest-pairing/pg_0001.pdf} 
}
\subfigure[Time-matching by $\Delta$]{
  \includegraphics[scale=0.43]{figs/earliest-pairing/pg_0004.pdf} 
  \label{fig:regression:by_X2}
}
\\
\subfigure[Within-bout time-matching $\Delta < 0$]{
  \includegraphics[scale=0.43]{figs/earliest-pairing/pg_0003.pdf}
  \label{fig:regression:within_bout}
}
\subfigure[Delayed time-matching $\Delta \geq 0$]{
  \includegraphics[scale=0.43]{figs/earliest-pairing/pg_0002.pdf}
  \label{fig:regression:delayed}
}
\end{center}

\caption{Longitudinal time-matching in minutes, showing: overall time-matching
(top-left), color-coded according to delay (top-right),  
within-bout only (bottom-left) and delayed only 
(bottom-right).
Each point on the scatter-plots below represents a pair of grooming events for 
a single dyad $\{A, B\}$.  The x-axis represents the number of minutes that $A$ 
spent grooming $B$, and the y-axis represents the time invested by $B$ in 
grooming $A$. \label{fig:regression}
}


% In order to detect possible over-representation, the inset 
% bar-charts show the proportion of total time recorded from each dyad.   The plots in the 
% upper-half of the panel show the entire dataset; in the top-left panel each 
% pair of events is  color-coded according to the dyad's frequency in the dataset, 
% whereas in the top-right the same data are color-coded according to the delay 
% in reciprocation $\Delta$.  The bottom panels separate the data-set according to the 
% delay $\Delta$; in the bottom-left we show within-bout time-matching ($\Delta < 0$), 
% whereas in the bottom-right we show delayed time-matching ($\Delta \geq 0$).  
%}
\end{figure*}

\begin{figure*}
 \begin{center}
  \includegraphics{figs/earliest-pairing/grooming-distribution-crop.pdf}
 \end{center}
 \caption{Distribution of total grooming duration over dyads.  The dyads in the 
top five percentiles ($\{AL, WH\}$,
$\{ BO , RO\} $,
$\{ DL , SA\} $,
$\{ HI , HP\} $,
$\{ KI , SR\} $,
$\{ KK , NI\} $,
$\{ KY , SA\} $,
$\{ SA , WI\} $) are highlighted in red.\label{fig:grooming-distribution}}
\end{figure*}

\begin{figure*}
  \begin{center}
    \subfigure[Null model ($\bar{R^2} = 0.07)$]{
      \includegraphics[scale=0.39]{figs/earliest-pairing/null-model-scatter-plot-crop.pdf}
      \label{fig:windowed:null}
    }
    \subfigure[Entire dataset ($\bar{R^2} = 0.55$)]{
      \includegraphics[scale=0.39]{figs/earliest-pairing/windowed-scatter-plot-all-crop.pdf}
      \label{fig:windowed:all}
    } \\
    \subfigure[Within-bout $\Delta < 0$ ($\bar{R^2} = 0.79$)]{
      \includegraphics[scale=0.39]{figs/earliest-pairing/windowed-scatter-plot-immediate-crop.pdf}
      \label{fig:windowed:immediate}
    } 
    \subfigure[Delayed $\Delta \geq 0$ ($\bar{R^2} = 0.08$)]{
      \includegraphics[scale=0.39]{figs/earliest-pairing/windowed-scatter-plot-delayed-crop.pdf}
      \label{fig:windowed:delayed}
    }   
  \end{center}
  \caption{Windowed time-matching.  The above plots illustrate 
time-matching when grooming durations are summed over time windows of 20 
minutes, 40 minutes, 1 hour and 4 hours.  The $\bar{R^2}$ values in parentheses in the caption beneath each figure shows the average of the $R^2$ values over each regression within the comparison group. Plot (a) shows windowed time-matching
of a null model in which grooming durations for each animal are independent and 
identically-distributed random variables.  Plot (b) shows
the empirical summed durations without distinguishing between within-bout or extra-bout
reciprocation.  When we separate the data according to the delay $\Delta$ we 
see that most of this time-matching is accounted for by within-bout activity (c).
When we restrict attention to delayed time-matching, the effect
largely disappears (d).\label{fig:windowed}}
\end{figure*}

\section{Reciprocity over different delay periods}

Michio Nakamura also suggested that although reciprocity might not occur after a 
very long delay due to memory decay, if we restrict attention to shorter delays 
then we might find evidence of time-matching.  We examined this question in a 
previous version of the manuscript, and we present the results here.  We used a 
20-minute moving time-window over $\Delta$, and perform time-matching regressions on data within the data, showing the regression slope (Figure~\ref{fig:windowed-regression-slope}) and fit (Figure~\ref{fig:windowed-regression-r-squared}). 
As can be seen, there is no evidence of time-matching once the window excludes
data from $\Delta \leq 0$.


\begin{figure}[h]
 \includegraphics[scale=0.8]{figs/windowed-regressions/pg_0002.pdf}
 \caption{Time-matching regression results for a 20 minute moving window of
 $\Delta$ showing slope and associate confidence intervals as the delay
 changes.  Results are not statistically significant once the window
 moves beyond the 0-20 minute period. \label{fig:windowed-regression-slope}}
\end{figure}

\begin{figure}[h]
 \includegraphics[scale=0.8]{figs/windowed-regressions/pg_0003.pdf}
 \caption{Time-matching regression results for a 20 minute moving window of
 $\Delta$ showing $R^2$.  The fit becomes very poor once
 the window moves beyond the 0-20 minute period. \label{fig:windowed-regression-r-squared}}
\end{figure}


\end{document}

